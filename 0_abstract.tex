\abstract{
\acg{[first draft]}

Diffusion has been around a while. Einstein modelled it as a random walk where everyone was independent from each other. This works pretty well and is used in everything from chemical reactions to light scattering. However, sometimes it sucks, like when you want to know something about the extremes instead of the average. Other methods have potential to better describe the extremes, like this RWRE stuff. This work seeks to measure extremes in diffusion numerically and experimentally. We make numerical measurements of the furthest particle, with whatever the results were. Then first passage time measurements of photons, which show failure of einstein random walk-based models and reveal a need for a better model to describe the behavior and capture the extremes. Finally we discuss proof of concept of an experiment to measure first passage times of diffusing colloids.

This dissertation includes previously published and unpublished coauthored material.
}
