\chapter{Conclusion}
\acg{first draft}

The work presented in this dissertation sheds new light on how the traditional random walk model, where each particle in a diffusing system takes independent and random steps through time, fails to capture the full picture of diffusion. While it works to describe the average particle well enough that we can use it to make predictions and discover new things across a wide range of disciplines, it does not accurately describe the behavior of extreme particles. This work revealed \acg{[first thing]}, proposed an experimental method for measuring extreme first passage times in diffusing colloids, and revealed the failure of random-walk-based models for photon propagation.

In Chapter \ref{ch2_1Drandom}, I built on existing literature on extreme particle locations which propose a connection between large deviations for times of order $\log(N)$ and the KPZ universality class~\cite{corwin_kardar_2012,quastel_one-dimensional_2015,kardar_dynamic_1986}, as well as a phase transition around times of $(\log(N))^2$~\cite{le_doussal_diffusion_2017}. A mathematical framework for a random walk in a random environment, where a one-dimensional system of particles undergo a random walk on a spatially and temporally variant random biasing field with biases drawn from a Beta distribution, had been proposed, and predicted a Tracy-Widom distribution for walker positions \acg{[I don't think this is quite accurate, need to revisit paper]}.

To determine what is measured in a system of particles diffusing in the aforementioned biased environment, Jacob Hass and I ran simulations of diffusing particles up to $N\approx10^{300}$ \acg{[$=$?]} in Chapter \ref{ch2_1Drandom}. We evolved the probability distribution function according to this biased walk, and collected extreme particle locations throughout time. By studying the statistics of these extreme particle locations, particularly the variance, we found the anticipated Tracy-Widom dynamics and uncovered a new phase with fluctuations tied to the KPZ universality class. A key result of this work is that the environmental variance of a physical system can be indirectly measured by determining the variance in the extreme particle location, then subtracting off the sample variance. In doing so, the underlying environment of this physical system can be better understood, and new insights into diffusion in real systems revealed.

Because we know from Chapter \ref{ch2_1Drandom} (and further work done by Hass et al~\cite{hass_first-passage_2024}) that we can learn about the shared environment of diffusing particles through the extreme value statistics, there is an opportunity to measure these statistics in a real physical system to learn about the underlying environment. In doing so, we can also confirm whether correlations in the environment are well represented by the random walk in a random environment model, rather than the traditional uncorrelated Einstein model.

In Chapter \ref{ch3_extras}, I developed an experimental apparatus to gather first passage times of colloids diffusing through water. This will allow future researchers to explore the first passage time statistics and uncover properties of the underlying environment in which these particles diffuse. I built a centrifuge that doubles as a microscope stage, isolating colloids to one end of a glass capillary tube, and tracking the first particle to enter and cross the microscope's field of view. While work remains to be done on ensuring that all colloids are truly isolated and that no flow is present in the capillaries, preliminary results are promising, and once these issues are resolved we should be able to collect first passage times and measure the statistics.

Photon transport has been well modelled by the Boltzmann Radiative Transfer Equation, but it's computationally intensive and, to our knowledge, has yet to be solved analytically. As a result, approximations to the equation are made. Assuming a lot of scatterers and minimal absorption, as well as long overall distances, we can approximate with a diffusion or telegraph equation. These models work quite well to predict the average photon behavior, as exhibited by how the models are used to measure optical parameters in a wide variety of applications. However, these models rely on the assumption that photon scattering is completely independent and totally elastic. Again, by neglecting correlated movement due to quantum mechanics and/or optical interference, we predict that these models fail the extremes.

In Chapter \ref{ch4_photons}, I measured photon extreme first passage times in random media of varying scatterer concentrations. From these measurements, I determined that the traditional diffusion approximation and telegraph equation fail to describe these extremes. Instead, the extreme photons travel as though they do not scatter at all, ``seeing'' an index-averaged medium well beyond the point where scatterer concentration are expected to delay the extreme first passage time. This shows that these models do not accurately capture the underlying physics, and a better model that factors in quantum mechanical or near-field effects is needed to more accurately describe extreme first passage photons~\cite{pattelli_role_2018,pini_non-self-similar_2024}. With a more precise model, we can better characterize the scattering environment photons travel through, with broad applications across scientific research~\cite{allgaier_diffuse_2021,amendola_accuracy_2021,taitelbaum_diagnosis_1999,bohren_absorption_1983}. 

Moving forward, measuring how the variance of the photon extreme first passage time distribution changes with scatterer concentration and laser power could connect this work to that of Chapter \ref{ch2_1Drandom} and \ref{ch3_extras}, particularly in determining if environmental properties can be extracted from these measurements. Additionally, changing the scattering medium, expanding the wavelength range in the experiment, varying polarization, and otherwise introducing new changing variables could provide further insight into the underlying physical phenomena. Numerically modelling the full pulse propagation, modelling diffusing photons as in Chapter \ref{ch2_1Drandom}, or using Monte Carlo methods to work with the full Boltzmann Radiative Tranfer Equation could all be fascinating ways to connect experimental data to theoretical models and numerical predictions beyond the two approximations explored in this work.

The work presented in this dissertation adds new insight to existing literature both on diffusive systems and photon transport. I measured extreme value statistics numerically and experimentally, revealing the need for models that account for correlated movement that exists either due to the shared medium (as in Chapter \ref{ch2_1Drandom}) or interference effects (as in Chapter \ref{ch3_photons}). These measurements have great potential to expose the limits of what we know about diffusive systems, and help pave the way to a better understanding of the underlying physics.