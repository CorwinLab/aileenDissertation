\chapter{Conclusion}
\acg{first draft}

The random walk model of diffusion has worked very well to describe the average particle, in applications from (one example) to photons (doing something else) to (third example). However, there are many instances in which it fails (cite them) (list them?). Recent mathematical work has proposed alternate models to account for these other cases, from the BC to quantum to whatever. This work measured the extreme value statistics in multiple systems and demonstrated the need for alternate models. First, we measured extreme first passage particle locations in numerical work and found the location to scale as (whatever). Second, we measured extreme first passage times in scattering photons and determined that the location of the peak of the extreme first passage time distribution is controlled by the averaged index of refraction of the scattering medium. Finally, we developed a proof of concept experimental setup to measure extreme first passage times in a system of diffusing silica microspheres. These experiments and simulations shed new light on the extreme value statistics of diffusing systems and how we can measure these statistics in real physical systems to learn about the underlying processes.

We know pretty well how to predict the average particle location in a system undergoing a random walk, especially an independent one. However, we expect particles nearby each other to move in a correlated manner. We can account for correlations for these particles via mathematical models that introduce bias into the random walk step distributions. Average behavior remains the same, but the predictions for the extreme particles differ greatly. Existing literature on extreme particle locations proposes a scaling of (whatever the Lawley papers say). 

We ran simulations of diffusing particles in Chapter \ref{1Drandom}. From these simulations, we measured the location of the furthest particle. For the furthest particles, we found (whatever we found). This is impactful because (reasons). 

Photon transport has been well modelled by the Boltzmann Radiative Transfer Equation, but it's computationally intensive and has yet to be solved analytically (to my knowledge?). As a result, approximations to the equation are made. Assuming a lot of scatterers and minimal absorption, as well as long overall distances, we can approximate with a diffusion or telegraph equation. These work pretty well to predict the average photon behavior, as exhibited by how useful the models are in a wide variety of applications. However, these models rely on the assumption that photon scattering is completely independent and totally elastic. Again, by neglecting correlated movement due to quantum mechanics and/or optical interference, we predict that these models fail the extremes.

In Chapter \ref{photons}, we measured photon extreme first passage times in random media of varying scatterer concentrations. From these measurements, we determined that (what we determined). This is important because (reasons). 

Finally, we have been able to measure average diffusing particles quite well for quite a while. We can predict the average and extreme particle behavior via these random walk models, accounting for correlation via biases. However, we don't know what actually happens unless we make a measurement of the extreme value statistics in a real physical diffusive system.

In Chapter \ref{extras} developed an experimental apparatus to gather first passage times of colloids diffusing through water. This allows us to explore the first passage time statistcs and can reveal (conect to jacogb's fpt paper). 

The work in this dissertation adds new insight to existing literature both on diffusive systems and photon scattering. We measured extreme value statistics in both, and revealed the need for models that account for correlated movement that exists either due to the shared medium or interference effects. We have built an experimental apparatus to measure extreme first passage times in a physical system, and show how much these measurements can reveal about the limits of what we truly know.
