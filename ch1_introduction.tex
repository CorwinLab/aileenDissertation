\chapter{Introduction}
\acg{[first draft]}

Understanding diffusion, or the process by which stuff spreads out, has been thinked about a while now. A question that people seem bothered to answer. While the idea that heat is a form of motion has a long history, early in the 17th century, it became a real ``hot'' subject amongst philosophers and scientists~\cite{bacon_novum_1902,boyle_new_1660,halley_historical_1686,newton_vii_1997}. In the 18th century, fluid mechanics and the kinetic theory of gases became pretty popular, and thermal conduction becamed a stuff.~\cite{du_chatelet_dissertation_1744,bernoulli_hydrodynamica_1738,lomonosov_mikhail_1970}. With the birth of modern thermodynamics in the 19th century, scientists worked hard to understand the motion of both heat and particles, including gas molecules~\cite{rudolf_clausius_mechanical_1867,sadi_carnot_reflexions_1824,fourier_analytical_1878,joule_scientific_2011,maxwell_theory_1908,thomson_dynamical_1851}. Of particular note was the Boltzmann transport equation, which characterized out-of-equilibrium thermodynamic systems through statistical quantities~\cite{boltzmann_weitere_1872}. Experimental work furthering the understanding of diffusive processes was done by Thomas Graham, who created experiments to understand gas diffusion~\cite{graham_xxvii_1833}. Building on this work, Adolph Fick ran experiments measuring concentrations of salt in water, leading to the influential differential equations now known as Fick's laws of diffusion~\cite{fick_v_1855}.

The first known observations of the irregular, wiggly movement characteristic of small particles can be attributed to the work of Jan Ingenhousz, who described the motion of coal dust on the surface of alcohol in 1785~\cite{ingenhousz_new_1785}. In the following century, while scientists worked to theoretically understand and mathematically describe the processes by which all sorts of stuff spreads out, Robert Brown and other microscope aficionados were publishing observations of the seemingly random motion of diffusive particles, like the dust that floats through air and the motion of pollen on the surface of water, causing many scientists of the time to debate whether these particles were ``alive''~\cite{brown_xxvii_1828,brown_xxiv_1829,van_der_pas_discovery_1971}. Brown's descriptions of this motion became the prolific, and the name ``Brownian motion'' caught on. 

The foundational work for connecting diffusion models and observations began with Louis Bachelier's 1900 thesis, which proposed that the stock market can be modelled by a ``random walk'' where each step is random and unpredictable~\cite{bachelier_theorie_1900}. In 1905, Einstein used a random walk model to formulate a diffusion equation to describe Brownian motion ~\cite{einstein_uber_1905}. Marian von Smoluchowski concurrently characterized Brownian motion in this manner~\cite{von_smoluchowski_zur_1906}. The random walk model proposed that the visible diffusing particles exhibiting Brownian motion are moving through an invisible environment of molecules, which are constantly moving around. These molecules collide with the larger visible particles, knocking them around and causing what appears to be erratic motion. This motion causes the diffusing particles to undergo a random walk, where their behavior can be modelled by independent random ``steps’’. This distribution of steps has a variance controlled by a diffusion coefficient, which is dependent on both the diffusing particle and the environment.

Einstein derived the Brownian motion random walk model through the molecular-kinetic theory of heat, connecting it in the process the differential equation Fick had developed decades before, and relating the diffusion coefficient to many physical quantities which can be measured~\cite{einstein_uber_1905}. Jean Baptiste Perrin confirmed the validity of Einstein's and Smoluchowski's work~\cite{perrin_mouvement_1909}. The random walk model can be used to understand many processes dominated by diffusion of the bulk of the material, such as human travel, molecular motion in the extracellular space of the brain, and the movement of chemicals in the bloodstream~\cite{gonzalez_understanding_2008, nicholson_extracellular_1998, ursino_mathematical_1989, zhang_lattice_2019}. Diffusion has also been applied to imaging technology such as magnetic resonance imaging, providing a method that impacts numerous important areas in medicine, such as understanding brain function and injuries, visualizing tumors, and analyzing fetal growth \cite{le_bihan_mr_1986, le_bihan_diffusion_2014, wijman_prognostic_2009, low_diffusion-weighted_2007, han_assessment_2015, abdel_razek_apparent_2019}. By describing the average diffusing particle well, we can use the random walk model to gain valuable insight into these processes and many more.

While the average particle can be very informative, we often are more interested in the extremes: the particle furthest from the origin, or even how quickly that furthest particle from the origin travels to a given destination. These first passage particles are important for describing all sorts of biological processes and chemical reactions, such as fertilization of an egg, \acg{(blah blah other things)}~\cite{redner_guide_2001,polizzi_mean_2016}. In applications from stock market fluctuations, to immune responses, to menopause timing, we care about the extreme first passage particles~\cite{barney_first-passage-time_2017,zsurkis_first_2024,schuss_redundancy_2019,lawley_slowest_2023}. These extreme first passage particles, i.e. the particles furthest from the origin or those that travel the most distance in the least amount of time, are characterized by different statistics~\cite{lawley_distribution_2020,lawley_probabilistic_2020,lawley_universal_2020}. 

The random walk model treats every diffusing particle as though they are behaving completely independent of their neighbors. This seems to contradict what we would expect: particles close together share the same environment, and should move similarly relative to particles farther apart. Recent work sheds insight on this, proposing a biased random walk that factors in correlations from the shared environment~\cite{barraquand_random-walk_2017,barraquand_moderate_2020}. The question is, what do we see when we observe systems undergoing diffusion, both numerically and physically? We can create large quantities of diffusive systems computationally. Further, we can probe the extreme value statistics of real physical systems. Photons sent into random media scatter in a way that seems analogous to diffusion; so much so, that we can use a diffusion approximation to the Radiative Transfer Equation to model their behavior quite well. Additionally, micron-sized silica beads suspended in water display the same thermal motions that Brown observed in pollen grains almost 200 years ago. We can track suspensions of these beads under a microscope and look for the extremes -- the first bead that travels a given distance from the origin -- and build up statistics about these particles.

The following three chapters are two papers which I composed during my time as a graduate student, and an additional chapter on unpublished work. Chapter \ref{1Drandom} (published in PRE~\cite{hass_anomalous_2023}) explores a method that I developed along with Eric Corwin and Jacob Hass. This allowed us to numerically simulate diffusion systems, and measure the location of the farthest particle from the origin. \acg{We found whatever scaling that matched whatever expectations}. In Chapter \ref{photons} (in review \acg{(cite)}), we used an ultrafast pulsed laser to send bursts of photons into a tube filled with silica nanospheres, timing the first photon to exit the tube, and build histograms of extreme first passage times. We measured the peak of this distribution, and determined that neither the diffusion nor telegraph model for photon transport correctly predicts the peak location; correlations cause more photons to travel directly through the tube as though they see an index-averaged medium. In chapter \ref{extras}, we developed an experimental apparatus that doubles as a centrifuge and a microscope stage in order to observe diffusion in colloidal suspensions, and we report on current design and the potential for future work.