\chapter{Introduction}
\acg{[first draft]}

Diffusion goes all the way back to Fick, Brown, etc. Tell a story about the history (briefly). Einstein et al put descriptions to the observations, connecting stat mech to observables. Describe the jist of what he and others did, einstein-schmoluchowski relation, etc. This is useful because of (reasons), and describes averages really well, so we use it a lot.

The average isn't always what really matters. Sometimes it's more important to talk about extremes - first particles, farthest particles, like in (stuff). The extremes are driven by different dynamics. Does the random walk work for extremes? Work suggests no, that biased environment blah blah works better, that we need to model diffusion differently to capture the full dynamic.

Let's figure out what happens numerically, look at real physical systems, photons and colloids, and see what happens there. Numerical simulations important because current work had just done (whatever). Photons important because we understand the bulk well and use it a lot, but work only looks at first passage distributions and extremes have been unmeasured (blah). Photons are really easy to produce with a laser, and we can use existing detection equipment to make a really straightforward system to measure extreme first passage times really fast bc light fast, adding a novel measurement to what we know about photons. A final system is colloids, which gives us a system on the scale of the stuff of Brown, and is useful because (whatever). We can use a microscope and design an experimental apparatus that spins down colloids and then watches them diffuse, 16 systems at a time, over and over to build first passage time statistics.

The following three chapters are two papers which I composed during my time as a graduate student, and an additional chapter on unpublished work. Chapter \ref{1Drandom} (published in PRE (citation)) explores a method that I developed along with Eric Corwin and Jacob Hass. This allowed us to measure stuff. We concluded conclusions. In Chapter \ref{photons} (in review (cite)), we used whatever to test whatever. The results are resulting. In chapter \ref{extras}, we did more stuff that has not led to publication yet but I'm sure I can talk about it still.
